\documentclass{article}

% set font encoding for PDFLaTeX or XeLaTeX
\usepackage{ifxetex}
\ifxetex
  \usepackage{fontspec}
\else
  \usepackage[T1]{fontenc}
  \usepackage[utf8]{inputenc}
  \usepackage{a4wide}
  \usepackage{lmodern}
  \usepackage[frenchb]{babel}
  \usepackage{amsmath,amsfonts,amssymb,amsthm,epsfig,epstopdf,titling,url,array,amssymb}
  \usepackage{graphicx}
  \usepackage{caption} 
  \usepackage{array}
  \usepackage{graphics,graphicx}
  \usepackage[usenames,dvipsnames]{pstricks}
  \usepackage{calc}
  \usepackage{multirow}
  \usepackage{algorithmic}
  \usepackage{algorithm}
  \usepackage{appendix}
  \usepackage{stmaryrd}
  \usepackage{tikz}  
  \usetikzlibrary{decorations.pathmorphing}
  \usetikzlibrary{decorations.pathreplacing}
  \usetikzlibrary{decorations.shapes}
  \usetikzlibrary{decorations.text}
  \usetikzlibrary{decorations.markings}
  \usetikzlibrary{decorations.footprints}
  \usepackage{color}
  \usepackage{geometry}
	\geometry{hmargin=2.5cm,vmargin=2cm}
  \usepackage{varioref}
  \usepackage{listings}
%  \usepackage[obeyspaces]{url}


  \lstdefinelanguage{Sage}[]{Python}
  {morekeywords={False,sage,True},sensitive=true}
	\lstset{frame=none,
  showtabs=False,
  showspaces=False,
  showstringspaces=False,
  commentstyle={\ttfamily\color{dgreencolor}},
  keywordstyle={\ttfamily\color{dbluecolor}\bfseries},
  stringstyle={\ttfamily\color{dgraycolor}\bfseries},
  language=Sage,
  basicstyle={\fontsize{10pt}{10pt}\ttfamily},
  aboveskip=0.3em,
  belowskip=0.1em,
  numbers=left,
  numberstyle=\footnotesize,
}
\definecolor{dblackcolor}{rgb}{0.0,0.0,0.0}
\definecolor{dbluecolor}{rgb}{0.01,0.02,0.7}
\definecolor{dgreencolor}{rgb}{0.2,0.4,0.0}
\definecolor{dgraycolor}{rgb}{0.30,0.3,0.30}
\newcommand{\dblue}{\color{dbluecolor}\bf}
\newcommand{\dred}{\color{dredcolor}\bf}
\newcommand{\dblack}{\color{dblackcolor}\bf}

\fi

% $\genfrac(){}{0}{a}{b}$


% used in maketitle
\title{Compter les points sur une courbe elliptique}
\author{Jérémie Coulaud}

% Enable SageTeX to run SageMath code right inside this LaTeX file.
% documentation: http://mirrors.ctan.org/macros/latex/contrib/sagetex/sagetexpackage.pdf
%\usepackage{sagetex}

\graphicspath{{../pictures/}}

\begin{document}
\newtheorem{prop}{Proposition}
\newtheorem{defi}{Définition}
\newtheorem{thm}{Théorème}
\maketitle
\newpage
\tableofcontents
\newpage

\section{Introduction aux courbes elliptiques}

Maintenant que l'on dispose de formule d'additions pour deux points sur une courbe elliptique on peut donner un sens à la multiplication scalaire d'un point comme $kP = \underbrace{P + \ldots + P}_{k \text{ fois}}$. On va noter cette multiplication scalaire par :
\newline

$\begin{array}{ccccc}
[l]_E & : & E(\mathbb{K}) & \to & E(\mathbb{K}) \\
 & & P & \mapsto & lP\\
\end{array}$

Ce qui nous permet de définir l'ensemble des points de l-torsion comme le noyau de $[l]$. On note $E[l]$ cet ensemble.
\newline
$$E[l] = \left\{ P \in E(\overline{\mathbb{K}}) \, | \, [l]P = 0_E \right\} $$

\section{Compter les points sur une courbe}
On va considérer dans la suite que la caractéristique du corps utilisée pour définir nos courbes elliptiques est plus grande que $3$. On peut donc écrire notre courbe elliptique sur $\mathbb{F}_p$ sous leur forme réduite $y^2 = x^3 + ax+b$

\subsection{Algorithme naif}
On note $E: y^2 = f(x)$, compter les points de $E$ revient donc pour chaque valeur de $x \in \mathbb{F}_p$ à regarder si $f(x)$ est un carré modulo $p$. On calcule donc le symbole de Legendre de $f(x)$, on a les cas suivant : 
\begin{itemize}
\item  $\genfrac(){}{0}{f(x)}{p} = -1$, $f(x)$ n'est pas un carré modulo $p$, on ne trouve aucun point appartenant à la courbe.
\item $\genfrac(){}{0}{f(x)}{p} = 0$, $f(x)$ est divisible par $p$, on trouve $1$ point sur la courbe.
\item $\genfrac(){}{0}{f(x)}{p} = 1$, $f(x)$ est un carré modulo $p$, on trouve $2$ points sur la courbe.
\end{itemize}
\medskip
Au final en considérant le point à l'infini on peut calculer le nombre de points de $E$ : 
\begin{equation*}
\#E(\mathbb{F}_p) = 1 + \sum_{x \in \mathbb{F}_p}(\genfrac(){}{0}{f(x)}{p} + 1)
\end{equation*}
Soit : 
\begin{equation}
\#E(\mathbb{F}_p) = 1 + p +\sum_{x \in \mathbb{F}_p}\genfrac(){}{0}{f(x)}{p}
\end{equation}
La complexité est en la taille de $p$. Cette méthode est pratique quand $p$ est petit mais devient impraticable s'il est trop grand.
\subsection{Shanks}
Il s'agit d'un algorithme Baby steps-giant steps de complexité exponentielle.

\subsection{Schoof}
Soit $E$ une courbe elliptique défini sur $\mathbb{F}_p$ avec $p$ premier $>3$ sous sa forme réduite 
$$ E: y^2 = x^3 + ax+b$$
On rappelle le théorème de Hasse-Weil:

\begin{thm}
$\#E(\mathbb{F}_p) = p + 1 - t$ avec $|t| \leq 2 \sqrt{p}$ trace de l'endomorphisme de Frobenius de $E$.
\end{thm}
Pour trouver le nombre de points de $E$ il faut donc déterminer $t$. 
L'idée de Schoof est de calculer $t$ modulo de petits nombres premiers puis d'utiliser le théorème des restes chinois. 

Avant de développer l'algorithme il est nécessaire de donner d'autres définitions. 

\begin{defi}[Frobenius]
Soit $E$ une courbe elliptique défini sur $\mathbb{F}_p$, l'endomorphisme de Frobenius est défini par 

$\begin{array}{ccccc}
\phi_p & : & E(\mathbb{F}_p) & \to & E(\mathbb{F}_p) \\
 & & (x,y) & \mapsto & (x^p, y^p) \\
\end{array}$
\end{defi}
On peut définir le polynôme caractéristique de cet endomorphisme par $\phi_p^2 - t \phi_p + p = 0$, cette relation reste vrai sur les points de $l$-torsion. Ainsi nous avons : 
\begin{equation}
\label{eqnfrobenius}
 \phi_p^2(P)  + [p_l]P = [t_{l}] \phi_p(P) \quad \forall P \in E[l]
\end{equation} 
Avec $t_l \equiv t \pmod l$, $p_l \equiv p \pmod l$ et $0 \leq t_l, p_l \leq l$.
Il faut aussi introduire les polynôme de division d'une courbe elliptiques $E$. On appelle $f_n(X)$ le n-ième polynôme de divisons de $E$.

\begin{defi}
Soit une courbe elliptique $E : y^2 = x^3 + ax+b$ défini sur $\mathbb{K}$. On définit $f_n(X)$ sur $\mathbb{Z}[x]$ de manière récursive : 


\begin{align*}
f_0(X) &= 0 \\
f_1(X) &= 1 \\
f_2(X) &= 1 \\
f_3(X) &= 3X^4 + 6aX^2 +12bX - a^2 \\
f_4(x) &= 2X^6 + 10aX^4 +40bX^3 - 10a^2X^2 - (a^2 + 8ab)X - 2(a^3 + 8b^2)
\end{align*}
On pose $F(X)= 4X^3 + 4aX + 4b$, et on a:

\begin{equation}
\left\lbrace
\begin{array}{ll}
f_{2n}& =  f_n(f_{n+2}f_{n-1}^2 - f_{n-2}f_{n+1}^2)   \\
f_{2n+1}& = \left\lbrace 
\begin{array}{ccc}
F^2f_{n+2}f_n^3 - f_{n-1}f_{n+1}^3 & \mbox{si} & m \text{ est pair}\\
f_{n+2}f_n^3 - f_{n-1}f_{n+1}^3F^2 & \mbox{si} & m \text{ est impair} \end{array}\right

\end{array}\right.
\end{equation} 
Ces polynômes sont de degrés au plus $\frac{(n^2 -1)}{2}$ si $n$ est pair, ou bien  au plus $\frac{(n^2 -2)}{2}$ si $n$ est impair.
\end{defi}

\begin{proof}
Preuve du degré de $f_n$ ? 
Avec $n$ premier on a $E[n] \simeq \frac{\mathbb{Z}}{n\mathbb{Z}} \times \frac{\mathbb{Z}}{n\mathbb{Z}}$, soit $n^2 - 1$ points de n-torsion ? on doit les compter 2fois donc j'imagine. Sinon la démo doit découler toute seule en utilisant les formules de recurrence mais un peu plus pénible a écrire
\end{proof}

On peut utiliser les polynôme de division pour calculer la multiplication scalaire d'un point de la courbe $E$.
On a les formules suivantes : 

\begin{thm}
Soit $E$ une courbe elliptique défini sur $\mathbb{K}$, un point $P$ sur cette courbe et $m \in \mathbb{N}^*$.

\begin{equation}
[m]P = 
\left\lbrace
\begin{array}{ccc}
O_E & \mbox{si} & P \in E[m]  \\
\left(    \frac{\phi_m(X,Y)}{\psi^2_m(X,Y)}, \frac{\omega_m(X,Y)}{\psi^3_m(X,Y)}\right) & \mbox{sinon}  & 
\end{array}\right.
\end{equation}

En posant : 
\begin{equation*}
\psi_m= \left\lbrace
\begin{array}{cc}
2Yf_m & \mbox{si m est pair} \\
f_m & \mbox{sinon}
\end{array}\right.
\end{equation*}
et 
\begin{equation*}
\left\lbrace
\begin{array}{ll}
\phi_m &= X \psi^2_m - \psi_{m-1}\psi_{m+1} \\
\psi_m \omega_m &= \psi_{2m}
\end{array}\right.
\end{equation*}
On peut aussi réécrire $[m]P$ sous cette forme : 
\begin{equation}\label{mP}
[m]P = \left\lbrace
\begin{array}{lll}
O_E & \mbox{si} & P \in E[m]  \\
\left(  X -   \frac{\psi_{m-1}(X,Y)\psi_{m+1}(X,Y)}{\psi^2_m(X,Y)}, \frac{\psi_{2m}(X,Y)}{\psi^4_m(X,Y)}\right) & \mbox{sinon}  & 
\end{array}\right.
\end{equation}
\end{thm}

\begin{proof}
On veut démontrer \ref{mP}. 
\newline
On note $[m]P = (x_1, y_1)$ 
On a alors : 


\begin{equation*}
x_1 = \frac{\phi_m}{\psi^2_m}\\
    =  \frac{X \psi^2_m - \psi_{m-1}\psi_{m+1}}{\psi_m^2} \\
    = X - \frac{\psi_{m-1}\psi_{m+1}}{\psi_m^2}
\end{equation*}
et 
\begin{equation*}
y_1 = \frac{\omega_m}{\psi^3_m} \\
	= \frac{\psi_{2m}}{\psi_m} \frac{1}{\psi_m^3} \\
	= \frac{\psi_{2m}}{\psi_m^4}
\end{equation*}
\end{proof}

On peut ainsi exprimer $[m]P$ comme un polynôme en $X,Y$. Mais ce n'est pas la seule particularité de ces polynôme utile pour notre algorithme. En effet $P = (x_1, y_1)$ est un point de l-torsion si et seulement si $x_1$ est une racine du l-ième polynôme de division $f_l$. De plus $P$ est sur la courbe $E$. Les points de l-torsion sont donc solution du système d'équation : 
\begin{equation}
E(X,Y) = Y^2 - X^3 - aX - b = 0, \quad f_l(X) = 0
\end{equation}
L'équation \ref{eqnfrobenius} peut donc se réécrire en utilisant les points de l-torsion. On va maintenant faire des calcul dans l'anneau $\mathbb{F}= \frac{\mathbb{F_p}[X,Y]}{(f_l(X), E(X,Y)}$.L'idée de l'algorithme de Schoof est donc de tester pour des valeurs $\tau_l \in \lbrace 0, \ldots, l-1 \rbrace$ si l'équation suivante est vrai:

\begin{equation}
(X^{p^2}, Y^{p^2}) + [p_l](X,Y) = [\tau_l](X^{p}, Y^{p})
\end{equation}
L'unique solution que l'on trouve est $t_l$.On répète l'opération pour d'autres $l$ premiers jusqu'avoir assez de $t_l$ pour appliquer le théorème des restes chinois et retrouver la valeur de $t$.
\newline
\medskip
On va maintenant détailler l'algorithme étape par étape. 
\end{document}