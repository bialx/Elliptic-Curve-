\documentclass{article}

% set font encoding for PDFLaTeX or XeLaTeX
\usepackage{ifxetex}
\ifxetex
  \usepackage{fontspec}
\else
  \usepackage[T1]{fontenc}
  \usepackage[utf8]{inputenc}
  \usepackage{a4wide}
  \usepackage{lmodern}
  \usepackage[frenchb]{babel}
  \usepackage{amsmath,amsfonts,amssymb,amsthm,epsfig,epstopdf,titling,url,array,amssymb}
  \usepackage{graphicx}
  \usepackage{caption} 
  \usepackage{array}
  \usepackage{graphics,graphicx}
  \usepackage[usenames,dvipsnames]{pstricks}
  \usepackage{calc}
  \usepackage{multirow}
  \usepackage{algorithmic}
  \usepackage{algorithm}
  \usepackage{appendix}
  \usepackage{stmaryrd}
  \usepackage{tikz}  
  \usetikzlibrary{decorations.pathmorphing}
  \usetikzlibrary{decorations.pathreplacing}
  \usetikzlibrary{decorations.shapes}
  \usetikzlibrary{decorations.text}
  \usetikzlibrary{decorations.markings}
  \usetikzlibrary{decorations.footprints}
  \usepackage{color}
  \usepackage{geometry}
	\geometry{hmargin=2.5cm,vmargin=2cm}
  \usepackage{varioref}
  \usepackage{listings}
%  \usepackage[obeyspaces]{url}


  \lstdefinelanguage{Sage}[]{Python}
  {morekeywords={False,sage,True},sensitive=true}
	\lstset{frame=none,
  showtabs=False,
  showspaces=False,
  showstringspaces=False,
  commentstyle={\ttfamily\color{dgreencolor}},
  keywordstyle={\ttfamily\color{dbluecolor}\bfseries},
  stringstyle={\ttfamily\color{dgraycolor}\bfseries},
  language=Sage,
  basicstyle={\fontsize{10pt}{10pt}\ttfamily},
  aboveskip=0.3em,
  belowskip=0.1em,
  numbers=left,
  numberstyle=\footnotesize,
}
\definecolor{dblackcolor}{rgb}{0.0,0.0,0.0}
\definecolor{dbluecolor}{rgb}{0.01,0.02,0.7}
\definecolor{dgreencolor}{rgb}{0.2,0.4,0.0}
\definecolor{dgraycolor}{rgb}{0.30,0.3,0.30}
\newcommand{\dblue}{\color{dbluecolor}\bf}
\newcommand{\dred}{\color{dredcolor}\bf}
\newcommand{\dblack}{\color{dblackcolor}\bf}

\fi

% $\genfrac(){}{0}{a}{b}$


% used in maketitle
\title{Compter les points sur une courbe elliptique}
\author{Jérémie Coulaud}

% Enable SageTeX to run SageMath code right inside this LaTeX file.
% documentation: http://mirrors.ctan.org/macros/latex/contrib/sagetex/sagetexpackage.pdf
%\usepackage{sagetex}

\graphicspath{{../pictures/}}

\begin{document}
\newtheorem{prop}{Proposition}
\newtheorem{defi}{Définition}
\newtheorem{thm}{Théorème}
\maketitle
\newpage
\tableofcontents
\newpage

\section{Introduction aux courbes elliptiques}


\section{Compter les points sur une courbe}
On va considérer dans la suite que la caractéristique du corps utilisée pour définir nos courbes elliptiques est plus grande que $3$. On peut donc écrire notre courbe elliptique sur $\mathbb{F}_p$ sous leur forme réduite $y^2 = x^3 + ax+b$

\subsection{Algorithme naif}
On note $E: y^2 = f(x)$, compter les points de $E$ revient donc pour chaque valeur de $x \in \mathbb{F}_p$ à regarder si $f(x)$ est un carré modulo $p$. On calcule donc le symbole de Legendre de $f(x)$, on a les cas suivant : 
\begin{itemize}
\item  $\genfrac(){}{0}{f(x)}{p} = -1$, $f(x)$ n'est pas un carré modulo $p$, on ne trouve aucun point appartenant à la courbe.
\item $\genfrac(){}{0}{f(x)}{p} = 0$, $f(x)$ est divisible par $p$, on trouve $1$ point sur la courbe.
\item $\genfrac(){}{0}{f(x)}{p} = 1$, $f(x)$ est un carré modulo $p$, on trouve $2$ points sur la courbe.
\end{itemize}
\medskip
Au final en considérant le point à l'infini on peut calculer le nombre de points de $E$ : 
\begin{equation*}
\#E(\mathbb{F}_p) = 1 + \sum_{x \in \mathbb{F}_p}(\genfrac(){}{0}{f(x)}{p} + 1)
\end{equation*}
Soit : 
\begin{equation}
\#E(\mathbb{F}_p) = 1 + p +\sum_{x \in \mathbb{F}_p}\genfrac(){}{0}{f(x)}{p}
\end{equation}
La complexité est en la taille de $p$. Cette méthode est pratique quand $p$ est petit mais devient impraticable s'il est trop grand.
\subsection{Shanks}
Il s'agit d'un algorithme Baby steps-giant steps de complexité exponentielle.

\subsection{Schoof}
Soit $E$ une courbe elliptique défini sur $\mathbb{F}_p$ avec $p$ premier $>3$ sous sa forme réduite 
$$ E: y^2 = x^3 + ax+b$$
On rappelle le théorème de Hasse-Weil:

\begin{thm}
$\#E(\mathbb{F}_p) = p + 1 - t$ avec $|t| \leq 2 \sqrt{p}$ trace de l'endomorphisme de Frobenius de $E$.
\end{thm}
Pour trouver le nombre de points de $E$ il faut donc déterminer $t$. 
L'idée de Schoof est de calculer $t$ modulo de petits nombres premiers puis d'utiliser le théorème des restes chinois. 

Avant de développer l'algorithme il est nécessaire de donner d'autres définitions. 

\begin{defi}[Frobenius]
Soit $E$ une courbe elliptique défini sur $\mathbb{F}_p$, l'endomorphisme de Frobenius est défini par 

$\begin{array}{ccccc}
\phi_p & : & E(\mathbb{F}_p) & \to & E(\mathbb{F}_p) \\
 & & (x,y) & \mapsto & (x^p, y^p) \\
\end{array}$
\end{defi}
On peut définir le polynôme caractéristique de cet endomorphisme par $\phi_p^2 - t \phi_p + p = 0$, cette relation reste vrai sur les points de $l$-torsion. Ainsi nous avons : 
$$  \phi_p^2(P)  + p_l = t_{l} \phi_p(P) \quad \forall P \in E[l]$$
Avec $t_l \equiv t \pmod l$, $p_l \equiv p \pmod l$ et $0 \leq t_l, p_l \leq l$.
 





\end{document}